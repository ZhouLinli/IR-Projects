% Options for packages loaded elsewhere
\PassOptionsToPackage{unicode}{hyperref}
\PassOptionsToPackage{hyphens}{url}
\PassOptionsToPackage{dvipsnames,svgnames,x11names}{xcolor}
%
\documentclass[
  letterpaper,
  DIV=11,
  numbers=noendperiod]{scrartcl}

\usepackage{amsmath,amssymb}
\usepackage{lmodern}
\usepackage{iftex}
\ifPDFTeX
  \usepackage[T1]{fontenc}
  \usepackage[utf8]{inputenc}
  \usepackage{textcomp} % provide euro and other symbols
\else % if luatex or xetex
  \usepackage{unicode-math}
  \defaultfontfeatures{Scale=MatchLowercase}
  \defaultfontfeatures[\rmfamily]{Ligatures=TeX,Scale=1}
\fi
% Use upquote if available, for straight quotes in verbatim environments
\IfFileExists{upquote.sty}{\usepackage{upquote}}{}
\IfFileExists{microtype.sty}{% use microtype if available
  \usepackage[]{microtype}
  \UseMicrotypeSet[protrusion]{basicmath} % disable protrusion for tt fonts
}{}
\makeatletter
\@ifundefined{KOMAClassName}{% if non-KOMA class
  \IfFileExists{parskip.sty}{%
    \usepackage{parskip}
  }{% else
    \setlength{\parindent}{0pt}
    \setlength{\parskip}{6pt plus 2pt minus 1pt}}
}{% if KOMA class
  \KOMAoptions{parskip=half}}
\makeatother
\usepackage{xcolor}
\setlength{\emergencystretch}{3em} % prevent overfull lines
\setcounter{secnumdepth}{-\maxdimen} % remove section numbering
% Make \paragraph and \subparagraph free-standing
\ifx\paragraph\undefined\else
  \let\oldparagraph\paragraph
  \renewcommand{\paragraph}[1]{\oldparagraph{#1}\mbox{}}
\fi
\ifx\subparagraph\undefined\else
  \let\oldsubparagraph\subparagraph
  \renewcommand{\subparagraph}[1]{\oldsubparagraph{#1}\mbox{}}
\fi


\providecommand{\tightlist}{%
  \setlength{\itemsep}{0pt}\setlength{\parskip}{0pt}}\usepackage{longtable,booktabs,array}
\usepackage{calc} % for calculating minipage widths
% Correct order of tables after \paragraph or \subparagraph
\usepackage{etoolbox}
\makeatletter
\patchcmd\longtable{\par}{\if@noskipsec\mbox{}\fi\par}{}{}
\makeatother
% Allow footnotes in longtable head/foot
\IfFileExists{footnotehyper.sty}{\usepackage{footnotehyper}}{\usepackage{footnote}}
\makesavenoteenv{longtable}
\usepackage{graphicx}
\makeatletter
\def\maxwidth{\ifdim\Gin@nat@width>\linewidth\linewidth\else\Gin@nat@width\fi}
\def\maxheight{\ifdim\Gin@nat@height>\textheight\textheight\else\Gin@nat@height\fi}
\makeatother
% Scale images if necessary, so that they will not overflow the page
% margins by default, and it is still possible to overwrite the defaults
% using explicit options in \includegraphics[width, height, ...]{}
\setkeys{Gin}{width=\maxwidth,height=\maxheight,keepaspectratio}
% Set default figure placement to htbp
\makeatletter
\def\fps@figure{htbp}
\makeatother

\usepackage{booktabs}
\usepackage{longtable}
\usepackage{array}
\usepackage{multirow}
\usepackage{wrapfig}
\usepackage{float}
\usepackage{colortbl}
\usepackage{pdflscape}
\usepackage{tabu}
\usepackage{threeparttable}
\usepackage{threeparttablex}
\usepackage[normalem]{ulem}
\usepackage{makecell}
\usepackage{xcolor}
\KOMAoption{captions}{tableheading}
\makeatletter
\makeatother
\makeatletter
\makeatother
\makeatletter
\@ifpackageloaded{caption}{}{\usepackage{caption}}
\AtBeginDocument{%
\ifdefined\contentsname
  \renewcommand*\contentsname{Table of contents}
\else
  \newcommand\contentsname{Table of contents}
\fi
\ifdefined\listfigurename
  \renewcommand*\listfigurename{List of Figures}
\else
  \newcommand\listfigurename{List of Figures}
\fi
\ifdefined\listtablename
  \renewcommand*\listtablename{List of Tables}
\else
  \newcommand\listtablename{List of Tables}
\fi
\ifdefined\figurename
  \renewcommand*\figurename{Figure}
\else
  \newcommand\figurename{Figure}
\fi
\ifdefined\tablename
  \renewcommand*\tablename{Table}
\else
  \newcommand\tablename{Table}
\fi
}
\@ifpackageloaded{float}{}{\usepackage{float}}
\floatstyle{ruled}
\@ifundefined{c@chapter}{\newfloat{codelisting}{h}{lop}}{\newfloat{codelisting}{h}{lop}[chapter]}
\floatname{codelisting}{Listing}
\newcommand*\listoflistings{\listof{codelisting}{List of Listings}}
\makeatother
\makeatletter
\@ifpackageloaded{caption}{}{\usepackage{caption}}
\@ifpackageloaded{subcaption}{}{\usepackage{subcaption}}
\makeatother
\makeatletter
\@ifpackageloaded{tcolorbox}{}{\usepackage[many]{tcolorbox}}
\makeatother
\makeatletter
\@ifundefined{shadecolor}{\definecolor{shadecolor}{rgb}{.97, .97, .97}}
\makeatother
\makeatletter
\makeatother
\ifLuaTeX
  \usepackage{selnolig}  % disable illegal ligatures
\fi
\IfFileExists{bookmark.sty}{\usepackage{bookmark}}{\usepackage{hyperref}}
\IfFileExists{xurl.sty}{\usepackage{xurl}}{} % add URL line breaks if available
\urlstyle{same} % disable monospaced font for URLs
\hypersetup{
  pdftitle={Academic Achievement Center (AAC) Courses Review},
  colorlinks=true,
  linkcolor={blue},
  filecolor={Maroon},
  citecolor={Blue},
  urlcolor={Blue},
  pdfcreator={LaTeX via pandoc}}

\title{Academic Achievement Center (AAC) Courses Review}
\author{}
\date{}

\begin{document}
\maketitle
\ifdefined\Shaded\renewenvironment{Shaded}{\begin{tcolorbox}[boxrule=0pt, interior hidden, frame hidden, borderline west={3pt}{0pt}{shadecolor}, sharp corners, enhanced, breakable]}{\end{tcolorbox}}\fi

\hypertarget{summary}{%
\section{Summary}\label{summary}}

\begin{itemize}
\tightlist
\item
  Among the past 5 years (2018 Fall to 2022 Spring), AAC 103 course has
  an average enrollment of 50 students.
\item
  Average pass rate for AAC 103 course is 82\%.
\item
  Students with higher than 2.0 semester GPA usually have higher pass
  rate in AAC103 than students with lower than 2.0 semester GPA.
\item
  An average of 73\% students in AAC 103 course have a higher than 2.0
  semester GPA.
\item
  Among those who pass AAC 103, 77\% have a higher than 2.0 GPA and 23\%
  do not have a good enough semester GPA. Among those who failed AAC
  103, 58\% still have a higher than 2.0 semester GPA (without
  succeeding in AAC 1O3 course).
\item
  Taking AAC 103 the second and third times does not improve pass rate
  or percentage of obtaining higher than 2.0 semester GPA.
\end{itemize}

\hypertarget{enrollment}{%
\section{Enrollment}\label{enrollment}}

\begin{table}

\caption{AAC course enrollment by term}
\centering
\fontsize{12}{14}\selectfont
\begin{tabular}[t]{>{\centering\arraybackslash}p{4em}>{\centering\arraybackslash}p{3em}>{\centering\arraybackslash}p{3em}>{\centering\arraybackslash}p{3em}>{\centering\arraybackslash}p{3em}>{\centering\arraybackslash}p{3em}>{\centering\arraybackslash}p{3em}>{\centering\arraybackslash}p{3em}>{\centering\arraybackslash}p{3em}>{\centering\arraybackslash}p{3em}}
\toprule
 & Average & 2018 Fall & 2019 Spring & 2019 Fall & 2020 Spring & 2020 Fall & 2021 Spring & 2021 Fall & 2022 Spring\\
\midrule
\cellcolor{white}{\textcolor{black}{\textbf{AAC102}}} & \cellcolor[HTML]{ffe099}{51} & \cellcolor{white}{} & \cellcolor[HTML]{FFCB4F}{76} & \cellcolor{white}{} & \cellcolor{white}{40} & \cellcolor{white}{} & \cellcolor{white}{} & \cellcolor{white}{} & \cellcolor{white}{37}\\
\cellcolor{white}{\textcolor{black}{\textbf{AAC103}}} & \cellcolor{white}{50} & \cellcolor{white}{50} & \cellcolor[HTML]{ffe099}{59} & \cellcolor[HTML]{ffe099}{53} & \cellcolor[HTML]{FFCB4F}{79} & \cellcolor{white}{38} & \cellcolor{white}{38} & \cellcolor{white}{43} & \cellcolor{white}{39}\\
\cellcolor{white}{\textcolor{black}{\textbf{AAC104}}} & \cellcolor{white}{34} & \cellcolor{white}{} & \cellcolor{white}{} & \cellcolor{white}{} & \cellcolor{white}{} & \cellcolor{white}{} & \cellcolor{white}{34} & \cellcolor{white}{} & \cellcolor{white}{}\\
\bottomrule
\end{tabular}
\end{table}

Among the past 5 years (2018 Fall to 2022 Spring), AAC courses
enrollment range from 34 to 79. An average enrollment for AAC103 is 50
students.

\hypertarget{pass-rate}{%
\section{Pass Rate}\label{pass-rate}}

\begin{table}

\caption{AAC courses pass rates}
\centering
\fontsize{12}{14}\selectfont
\begin{tabular}[t]{>{\centering\arraybackslash}p{4em}>{\centering\arraybackslash}p{3em}>{\centering\arraybackslash}p{3em}>{\centering\arraybackslash}p{3em}>{\centering\arraybackslash}p{3em}>{\centering\arraybackslash}p{3em}>{\centering\arraybackslash}p{3em}>{\centering\arraybackslash}p{3em}>{\centering\arraybackslash}p{3em}>{\centering\arraybackslash}p{3em}}
\toprule
 & Average & 2018 Fall & 2019 Spring & 2019 Fall & 2020 Spring & 2020 Fall & 2021 Spring & 2021 Fall & 2022 Spring\\
\midrule
\cellcolor{white}{\textcolor{black}{\textbf{AAC102}}} & \cellcolor[HTML]{ffe099}{\textcolor{black}{88\%}} & \cellcolor{white}{\textcolor{black}{}} & \cellcolor{white}{\textcolor{black}{66\%}} & \cellcolor{white}{\textcolor{black}{}} & \cellcolor[HTML]{FFCB4F}{\textcolor{black}{100\%}} & \cellcolor{white}{\textcolor{black}{}} & \cellcolor{white}{\textcolor{black}{}} & \cellcolor{white}{\textcolor{black}{}} & \cellcolor[HTML]{FFCB4F}{\textcolor{black}{97\%}}\\
\cellcolor{white}{\textcolor{black}{\textbf{AAC103}}} & \cellcolor[HTML]{ffe099}{\textcolor{black}{82\%}} & \cellcolor{white}{\textcolor{black}{74\%}} & \cellcolor[HTML]{ffe099}{\textcolor{black}{83\%}} & \cellcolor[HTML]{ffe099}{\textcolor{black}{83\%}} & \cellcolor[HTML]{ffe099}{\textcolor{black}{81\%}} & \cellcolor{white}{\textcolor{black}{76\%}} & \cellcolor[HTML]{ffe099}{\textcolor{black}{84\%}} & \cellcolor[HTML]{FFCB4F}{\textcolor{black}{93\%}} & \cellcolor{white}{\textcolor{black}{79\%}}\\
\cellcolor{white}{\textcolor{black}{\textbf{AAC104}}} & \cellcolor[HTML]{ffe099}{\textcolor{black}{85\%}} & \cellcolor{white}{\textcolor{black}{}} & \cellcolor{white}{\textcolor{black}{}} & \cellcolor{white}{\textcolor{black}{}} & \cellcolor{white}{\textcolor{black}{}} & \cellcolor{white}{\textcolor{black}{}} & \cellcolor[HTML]{ffe099}{\textcolor{black}{85\%}} & \cellcolor{white}{\textcolor{black}{}} & \cellcolor{white}{\textcolor{black}{}}\\
\bottomrule
\end{tabular}
\end{table}

Average pass rate for AAC courses are above 82\% across the past 5
academic years.

\hypertarget{aac103-pass-rate-for-students-with-different-semester-gpa}{%
\subsection{AAC103 Pass rate for students with different semester
GPA}\label{aac103-pass-rate-for-students-with-different-semester-gpa}}

\begin{table}

\caption{AAC103 pass rates by semester GPA}
\centering
\fontsize{12}{14}\selectfont
\begin{tabular}[t]{>{\centering\arraybackslash}p{7em}>{\centering\arraybackslash}p{4em}>{\centering\arraybackslash}p{4em}>{\centering\arraybackslash}p{4em}}
\toprule
 & Semester GPA >2 & Semester GPA <2 & Difference\\
\midrule
\cellcolor{white}{\textcolor{black}{2018 Fall}} & \cellcolor{white}{\textcolor{black}{76\%}} & \cellcolor{white}{\textcolor{black}{67\%}} & \cellcolor{white}{9\%}\\
\cellcolor{white}{\textcolor{black}{2019 Fall}} & \cellcolor[HTML]{ffe099}{\textcolor{black}{86\%}} & \cellcolor{white}{\textcolor{black}{73\%}} & \cellcolor{white}{13\%}\\
\cellcolor{white}{\textcolor{black}{2019 Spring}} & \cellcolor[HTML]{ffe099}{\textcolor{black}{87\%}} & \cellcolor{white}{\textcolor{black}{75\%}} & \cellcolor{white}{12\%}\\
\cellcolor{white}{\textcolor{black}{2020 Fall}} & \cellcolor[HTML]{ffe099}{\textcolor{black}{88\%}} & \cellcolor{white}{\textcolor{black}{54\%}} & \cellcolor[HTML]{aaaaaa}{34\%}\\
\cellcolor{white}{\textcolor{black}{2020 Spring}} & \cellcolor[HTML]{ffe099}{\textcolor{black}{83\%}} & \cellcolor{white}{\textcolor{black}{74\%}} & \cellcolor{white}{10\%}\\
\addlinespace
\cellcolor{white}{\textcolor{black}{2021 Fall}} & \cellcolor[HTML]{FFCB4F}{\textcolor{black}{93\%}} & \cellcolor[HTML]{FFCB4F}{\textcolor{black}{93\%}} & \cellcolor{white}{0\%}\\
\cellcolor{white}{\textcolor{black}{2021 Spring}} & \cellcolor[HTML]{ffe099}{\textcolor{black}{89\%}} & \cellcolor{white}{\textcolor{black}{70\%}} & \cellcolor[HTML]{cccccc}{19\%}\\
\cellcolor{white}{\textcolor{black}{2022 Spring}} & \cellcolor[HTML]{ffe099}{\textcolor{black}{86\%}} & \cellcolor{white}{\textcolor{black}{64\%}} & \cellcolor[HTML]{cccccc}{22\%}\\
\bottomrule
\end{tabular}
\end{table}

Students with higher than 2.0 semester GPA usually have higher pass rate
in AAC103 than students with lower than 2.0 semester GPA. The difference
of AAC103 pass rate for students from different semester GPA is as large
as 34\% in 2020 Fall.

\hypertarget{semester-gpa}{%
\section{Semester GPA}\label{semester-gpa}}

\begin{table}

\caption{Percentage of higher than 2.0 semester GPA}
\centering
\fontsize{12}{14}\selectfont
\begin{tabular}[t]{>{\centering\arraybackslash}p{3em}>{\centering\arraybackslash}p{3em}>{\centering\arraybackslash}p{3em}>{\centering\arraybackslash}p{3em}>{\centering\arraybackslash}p{3em}>{\centering\arraybackslash}p{3em}>{\centering\arraybackslash}p{3em}>{\centering\arraybackslash}p{3em}>{\centering\arraybackslash}p{3em}>{\centering\arraybackslash}p{3em}}
\toprule
 & Average & 2018 Fall & 2019 Spring & 2019 Fall & 2020 Spring & 2020 Fall & 2021 Spring & 2021 Fall & 2022 Spring\\
\midrule
\cellcolor{white}{\textcolor{black}{\textbf{AAC102}}} & \cellcolor[HTML]{FFCB4F}{\textcolor{black}{85\%}} & \cellcolor{white}{\textcolor{black}{}} & \cellcolor[HTML]{FFCB4F}{\textcolor{black}{88\%}} & \cellcolor{white}{\textcolor{black}{}} & \cellcolor[HTML]{FFCB4F}{\textcolor{black}{95\%}} & \cellcolor{white}{\textcolor{black}{}} & \cellcolor{white}{\textcolor{black}{}} & \cellcolor{white}{\textcolor{black}{}} & \cellcolor[HTML]{ffe099}{\textcolor{black}{73\%}}\\
\cellcolor{white}{\textcolor{black}{\textbf{AAC103}}} & \cellcolor[HTML]{ffe099}{\textcolor{black}{73\%}} & \cellcolor[HTML]{FFCB4F}{\textcolor{black}{82\%}} & \cellcolor{white}{\textcolor{black}{66\%}} & \cellcolor[HTML]{ffe099}{\textcolor{black}{79\%}} & \cellcolor[HTML]{ffe099}{\textcolor{black}{76\%}} & \cellcolor{white}{\textcolor{black}{66\%}} & \cellcolor[HTML]{ffe099}{\textcolor{black}{74\%}} & \cellcolor{white}{\textcolor{black}{67\%}} & \cellcolor[HTML]{ffe099}{\textcolor{black}{72\%}}\\
\cellcolor{white}{\textcolor{black}{\textbf{AAC104}}} & \cellcolor{white}{\textcolor{black}{44\%}} & \cellcolor{white}{\textcolor{black}{}} & \cellcolor{white}{\textcolor{black}{}} & \cellcolor{white}{\textcolor{black}{}} & \cellcolor{white}{\textcolor{black}{}} & \cellcolor{white}{\textcolor{black}{}} & \cellcolor{white}{\textcolor{black}{44\%}} & \cellcolor{white}{\textcolor{black}{}} & \cellcolor{white}{\textcolor{black}{}}\\
\bottomrule
\end{tabular}
\end{table}

An average of 85\% students in AAC 102 course have a higher than 2.0
semester GPA in the past 5 years. The average percentage of higher than
2.0 semester GPA for AAC 103 course is 73\%.

\hypertarget{semester-gpa-for-students-who-pass-or-fail-aac103}{%
\subsection{Semester GPA for students who pass or fail
AAC103}\label{semester-gpa-for-students-who-pass-or-fail-aac103}}

\begin{table}

\caption{Percentage of higher than 2.0 semester GPA among students who passed or failed AAC103}
\centering
\fontsize{12}{14}\selectfont
\begin{tabular}[t]{>{\centering\arraybackslash}p{2em}>{\centering\arraybackslash}p{3em}>{\centering\arraybackslash}p{3em}>{\centering\arraybackslash}p{3em}>{\centering\arraybackslash}p{3em}>{\centering\arraybackslash}p{3em}>{\centering\arraybackslash}p{3em}>{\centering\arraybackslash}p{3em}>{\centering\arraybackslash}p{3em}>{\centering\arraybackslash}p{3em}}
\toprule
 & Average & 2018 Fall & 2019 Spring & 2019 Fall & 2020 Spring & 2020 Fall & 2021 Spring & 2021 Fall & 2022 Spring\\
\midrule
\cellcolor{white}{\textcolor{black}{Fail}} & \cellcolor{white}{\textcolor{black}{58\%}} & \cellcolor[HTML]{ffe099}{\textcolor{black}{77\%}} & \cellcolor{white}{\textcolor{black}{50\%}} & \cellcolor{white}{\textcolor{black}{67\%}} & \cellcolor{white}{\textcolor{black}{67\%}} & \cellcolor{white}{\textcolor{black}{33\%}} & \cellcolor{white}{\textcolor{black}{50\%}} & \cellcolor{white}{\textcolor{black}{67\%}} & \cellcolor{white}{\textcolor{black}{50\%}}\\
\cellcolor{white}{\textcolor{black}{Pass}} & \cellcolor[HTML]{ffe099}{\textcolor{black}{77\%}} & \cellcolor[HTML]{FFCB4F}{\textcolor{black}{84\%}} & \cellcolor{white}{\textcolor{black}{69\%}} & \cellcolor[HTML]{FFCB4F}{\textcolor{black}{82\%}} & \cellcolor[HTML]{ffe099}{\textcolor{black}{78\%}} & \cellcolor[HTML]{ffe099}{\textcolor{black}{76\%}} & \cellcolor[HTML]{ffe099}{\textcolor{black}{78\%}} & \cellcolor{white}{\textcolor{black}{68\%}} & \cellcolor[HTML]{ffe099}{\textcolor{black}{77\%}}\\
\bottomrule
\end{tabular}
\end{table}

Among students who passed AAC 103, an average of 77\% students have a
higher than 2.0 semester GPA. In contrast, only an average of 58\%
students have a higher than 2.0 semester GPA among students who failed
AAC 103.

\hypertarget{retaking-aac103}{%
\section{Retaking AAC103}\label{retaking-aac103}}

In analysis, we found 4 students took AAC 103 4 times and 2 of them took
the course 5 times. Due to the small sample size, we exclude those
students from pass rate reviewing. The following tables reported
students who took AAC 103 for the 1st, 2nd, and 3rd time.

\hypertarget{pass-rate-when-retaking} & 66\\
\cellcolor{white}{\textcolor{black}{2nd}} & \cellcolor{white}{79\%} & 66\\
\cellcolor{white}{\textcolor{black}{3rd}} & \cellcolor{white}{77\%} & 13\\
\bottomrule
\end{tabular}
\end{table}

First time takers has the highest pass rate of 83\%. In other words, AAC
103's pass rate did not increase for along with students times of
retaking the course.

\hypertarget{semester-gpa-when-retaking} & 66\\
\cellcolor{white}{\textcolor{black}{2nd}} & \cellcolor[HTML]{ffe099}{73\%} & 66\\
\cellcolor{white}{\textcolor{black}{3rd}} & \cellcolor{white}{46\%} & 13\\
\bottomrule
\end{tabular}
\end{table}

Percentage of higher than 2.0 semester GPA among AAC 103 retakers range
from 46\%-73\%. The second time that students taking AAC 103 has 73\%,
the highest percentage, of higher than 2.0 GPA.



\end{document}
